\documentclass[10pt,a4paper,fleqn,landscape]{article}
\usepackage[ngerman]{babel}
\usepackage[utf8]{inputenc}
\usepackage[T1]{fontenc}

\usepackage[landscape]{geometry}
\usepackage{multicol}
% for landscape format: insert "landscape" at documentclass and geometry

\usepackage{float}
\usepackage{graphicx}
\usepackage{amsmath, amsfonts, amssymb,esint}
\usepackage{trfsigns}

\geometry{top=0.5cm,left=1cm,right=1cm,bottom=0.5cm}

\setlength{\mathindent}{0mm}
\setlength{\parindent}{0mm}
\setlength{\parskip}{0mm}
\setlength{\jot}{5pt}
\setlength{\abovedisplayskip}{0pt}
\setlength{\belowdisplayskip}{5pt}
\setlength{\abovedisplayshortskip}{0pt}
\setlength{\belowdisplayshortskip}{5pt}

\setlength{\columnseprule}{0.25pt}
\setlength{\premulticols}{1pt}
\setlength{\postmulticols}{1pt}
\setlength{\multicolsep}{1pt}
\setlength{\columnsep}{10pt}


\makeatletter
\renewcommand{\section}{\@startsection{section}{1}{0mm}%
                                {-1ex plus -.5ex minus -.2ex}%
                                {0.5ex plus .2ex}%x
                                {\normalfont\large\bfseries}}
\renewcommand{\subsection}{\@startsection{subsection}{2}{0mm}%
                                {-1explus -.5ex minus -.2ex}%
                                {0.5ex plus .2ex}%
                                {\normalfont\normalsize\bfseries}}
\renewcommand{\subsubsection}{\@startsection{subsubsection}{3}{0mm}%
                                {-1ex plus -.5ex minus -.2ex}%
                                {1ex plus .2ex}%
                                {\normalfont\small\bfseries}}

\g@addto@macro\normalsize{	% adapt if changing fontsize
  \setlength\abovedisplayskip{5pt}
  \setlength\belowdisplayskip{5pt}
  \setlength\abovedisplayshortskip{5pt}
  \setlength\belowdisplayshortskip{5pt}
}                   
                                
\makeatother
\setcounter{secnumdepth}{0}

% Command Defs ------------------------------------------------------------------------------------------------

% Command for horizontal separation line
\newcommand{\EW}{\mathbb{E}}\newcommand{\HL}{\rule{\linewidth}{0.25pt}}

\newcommand{\vc}[1]{\vec{\underline{#1}}} 		% <-- Use this shit for komplex vectors: 	example: \vc{H}
\renewcommand{\c}[1]{\underline{#1}}			% <-- Use this shit für komplexe scalars: 	example: \c{H}_x
\renewcommand{\j}{{\mathrm j}}					% imaginary unit
\newcommand{\D}{{\mathrm d}}					% differential d for dx and stuff

% -------------------------------------------------------------------------------------------------------------

\pagestyle{empty}

\begin{document}
	
%\small
\begin{multicols}{2}
\subsection{Zeitharmonische Maxwell-Gleichungen}
für anisotrope, homogene Medien
	\begin{align*}
	\nabla\times\vc{H} &= \vc{J} + \j\omega\varepsilon\vc{E} &\iff&& \oint_{\partial A}\vc{H}\cdot\D\vec{s} &= \iint_A\left( \vc{J} + \j\omega\varepsilon\vc{E}\right) \cdot\D\vec{A}\\
	-\nabla\times\vc{E} &= \vc{M} + \j\omega\mu\vc{H} &\iff&& -\oint_{\partial A}\vc{E}\cdot\D\vec{s} &= \iint_A\left( \vc{M} + \j\omega\mu\vc{H}\right) \cdot\D\vec{A}\\
	\nabla\cdot\vc{H}&=\c{\rho}_m/\mu &\iff&& \varoiint_{\partial V}\vc{H}\cdot\D\vec{A}&=\frac{1}{\mu}\iiint_V \c{\rho}_m\D V\\
	\nabla\cdot\vc{E}&=\c{\rho}_e/\varepsilon &\iff&& \varoiint_{\partial V}\vc{E}\cdot\D\vec{A}&=\frac{1}{\varepsilon}\iiint_V \c{\rho}_e\D V
	\end{align*}
für Freiraumausbreitung und allgemein Vakuum: $\c{\rho}_e=\c{\rho}_m=0$ und $\vc{J}=\vc{M}=\vec{0}$

Poynting-Vektor; Wirkleistung
	\begin{equation*}
	\vc{S}=\frac{1}{2}\vc{E}\times\vc{H}^*\quad;\quad P=\varoiint\Re\left\lbrace \vc{S} \right\rbrace \D\vec{A} =\frac{1}{2}\varoiint\Re\left\lbrace \vc{E}\times\vc{H}^* \right\rbrace \D\vec{A}
	\end{equation*}

\subsection{EM-Wellen: allgemeine Zusammenhänge}


\subsection*{Wellenleiter (konstanter Querschnitt)}
	\begin{equation*}
	\lambda=\frac{\lambda_0}{\sqrt{1-\left(\frac{\beta_c}{\beta_0 }\right)^2 }}=\frac{\lambda_0}{\sqrt{1-\left(\frac{f_c}{f}\right)^2 }}=\frac{\lambda_0}{\sqrt{1-\left(\frac{\lambda_0}{\lambda_c}\right)^2 }}
	\end{equation*}

\subsection*{Feldwellenwiderstand}
TEM-Wellen
	\begin{align*}
		\frac{\vc{E}_x}{\vc{H}_y} &= -\frac{\vc{E}_y}{\vc{H}_x} = \pm\frac{\omega\mu}{\beta} = \pm\sqrt{\frac{\mu}{\varepsilon}} = \pm Z_F \\
		Z_{F0} &= \sqrt{\frac{\mu_0}{\varepsilon_0}} = 120\pi\Omega = 377\Omega
	\end{align*}
TE-Wellen
	\begin{equation*}
		Z_{FH} = \pm\frac{\vc{E}_x}{\vc{H}_y} = \mp\frac{\vc{E}_y}{\vc{H}_x} = \frac{\omega\mu}{\beta} = \frac{\omega\mu}{\beta_0\sqrt{1-\left(\frac{\beta_c}{\beta_0}\right)^2}} = \frac{Z_F}{\sqrt{1-\left(\frac{\beta_c}{\beta_0}\right)^2}} = \frac{Z_F}{\sqrt{1-\left(\frac{f_c}{f_0}\right)^2}} = \frac{Z_F}{\sqrt{1-\left(\frac{\lambda_0}{\lambda_c}\right)^2}}
	\end{equation*}
TM-Wellen
	\begin{align*}
		Z_{FE} &= \pm\frac{\vc{E}_x}{\vc{H}_y} = \mp\frac{\vc{E}_y}{\vc{H}_x} = \frac{\beta}{\omega\varepsilon} = \frac{\beta_0\sqrt{1-\left(\frac{\beta_c}{\beta_0}\right)^2}}{\omega\varepsilon} \\
		&= Z_F \sqrt{1-\left(\frac{\beta_c}{\beta_0}\right)^2} = Z_F \sqrt{1-\left(\frac{f_c}{f_0}\right)^2} = Z_F \sqrt{1-\left(\frac{\lambda_0}{\lambda_c}\right)^2}
	\end{align*}
\end{multicols}	

\end{document}
